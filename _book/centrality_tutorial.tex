\documentclass[]{book}
\usepackage{lmodern}
\usepackage{amssymb,amsmath}
\usepackage{ifxetex,ifluatex}
\usepackage{fixltx2e} % provides \textsubscript
\ifnum 0\ifxetex 1\fi\ifluatex 1\fi=0 % if pdftex
  \usepackage[T1]{fontenc}
  \usepackage[utf8]{inputenc}
\else % if luatex or xelatex
  \ifxetex
    \usepackage{mathspec}
  \else
    \usepackage{fontspec}
  \fi
  \defaultfontfeatures{Ligatures=TeX,Scale=MatchLowercase}
\fi
% use upquote if available, for straight quotes in verbatim environments
\IfFileExists{upquote.sty}{\usepackage{upquote}}{}
% use microtype if available
\IfFileExists{microtype.sty}{%
\usepackage{microtype}
\UseMicrotypeSet[protrusion]{basicmath} % disable protrusion for tt fonts
}{}
\usepackage[margin=1in]{geometry}
\usepackage{hyperref}
\hypersetup{unicode=true,
            pdftitle={An Introduction to Network Centrality},
            pdfauthor={David Schoch},
            pdfborder={0 0 0},
            breaklinks=true}
\urlstyle{same}  % don't use monospace font for urls
\usepackage{natbib}
\bibliographystyle{apalike}
\usepackage{longtable,booktabs}
\usepackage{graphicx,grffile}
\makeatletter
\def\maxwidth{\ifdim\Gin@nat@width>\linewidth\linewidth\else\Gin@nat@width\fi}
\def\maxheight{\ifdim\Gin@nat@height>\textheight\textheight\else\Gin@nat@height\fi}
\makeatother
% Scale images if necessary, so that they will not overflow the page
% margins by default, and it is still possible to overwrite the defaults
% using explicit options in \includegraphics[width, height, ...]{}
\setkeys{Gin}{width=\maxwidth,height=\maxheight,keepaspectratio}
\IfFileExists{parskip.sty}{%
\usepackage{parskip}
}{% else
\setlength{\parindent}{0pt}
\setlength{\parskip}{6pt plus 2pt minus 1pt}
}
\setlength{\emergencystretch}{3em}  % prevent overfull lines
\providecommand{\tightlist}{%
  \setlength{\itemsep}{0pt}\setlength{\parskip}{0pt}}
\setcounter{secnumdepth}{5}
% Redefines (sub)paragraphs to behave more like sections
\ifx\paragraph\undefined\else
\let\oldparagraph\paragraph
\renewcommand{\paragraph}[1]{\oldparagraph{#1}\mbox{}}
\fi
\ifx\subparagraph\undefined\else
\let\oldsubparagraph\subparagraph
\renewcommand{\subparagraph}[1]{\oldsubparagraph{#1}\mbox{}}
\fi

%%% Use protect on footnotes to avoid problems with footnotes in titles
\let\rmarkdownfootnote\footnote%
\def\footnote{\protect\rmarkdownfootnote}

%%% Change title format to be more compact
\usepackage{titling}

% Create subtitle command for use in maketitle
\newcommand{\subtitle}[1]{
  \posttitle{
    \begin{center}\large#1\end{center}
    }
}

\setlength{\droptitle}{-2em}

  \title{An Introduction to Network Centrality}
    \pretitle{\vspace{\droptitle}\centering\huge}
  \posttitle{\par}
  \subtitle{with Applications in R}
  \author{David Schoch}
    \preauthor{\centering\large\emph}
  \postauthor{\par}
      \predate{\centering\large\emph}
  \postdate{\par}
    \date{2018-12-07}

\usepackage{booktabs}

\usepackage{amsthm}
\newtheorem{theorem}{Theorem}[chapter]
\newtheorem{lemma}{Lemma}[chapter]
\theoremstyle{definition}
\newtheorem{definition}{Definition}[chapter]
\newtheorem{corollary}{Corollary}[chapter]
\newtheorem{proposition}{Proposition}[chapter]
\theoremstyle{definition}
\newtheorem{example}{Example}[chapter]
\theoremstyle{definition}
\newtheorem{exercise}{Exercise}[chapter]
\theoremstyle{remark}
\newtheorem*{remark}{Remark}
\newtheorem*{solution}{Solution}
\begin{document}
\maketitle

{
\setcounter{tocdepth}{1}
\tableofcontents
}
\hypertarget{preface}{%
\chapter*{Preface}\label{preface}}
\addcontentsline{toc}{chapter}{Preface}

\hypertarget{intro}{%
\chapter{Introduction}\label{intro}}

\hypertarget{indices}{%
\chapter{Centrality Indices}\label{indices}}

The purpose of network centrality is to identify important actors or
entities in a network. Structural importance is determined by so called
\emph{measures of centrality}, commonly defined in terms of indices
\(c: V \to \mathbb{R}\) interpreted as \[
c(u) > c(v) \iff u \text{ is more central than } v.
\] Since the meaning of structural importance is by no means
unambiguous, a vast amount of different indices have been introduced
over time. In addition, any mapping \(c: V \to \mathbb{R}\) induces a
ranking of nodes, but not every such ranking might represent a plausible
concept of structural importance.

\hypertarget{degree}{%
\section{Degree}\label{degree}}

Degree centrality is the most simple form of a centrality index. It is
defined as \[
c_d(u) = \{v : \{u,v\} \in E\} = \lvert N(u) \rvert
\] Degree centrality is a purely \emph{local} measure since it only
depends on the direct neighborhood of a node. A simple application
example is popularity in friend- ship networks, i.e. ``who has the most
friends?''. The definition of degree centrality can easily be adapted
for directed and weighted networks. For directed networks, \[
c_d^+(u)=\{v : (u,v) \in E\} = \lvert N^+(u) \rvert \quad \text{and} \quad 
c_d^-(u)=\{v : (v,u) \in E\} = \lvert N^-(u) \rvert
\] are the \emph{out-degree} and \emph{in-degree}, respectively.
\emph{Weighted degree}, sometimes also refered to as \emph{strength}, is
defined as \[
c_{wd}(u)=\sum_{v \in N(u)} w_{uv}.
\]

\hypertarget{betweenness-and-variants}{%
\section{Betweenness (and variants)}\label{betweenness-and-variants}}

Betweenness was independently developed by \citet{a-rdg-71} and
\citet{f-smcbb-77} and is an extension of \emph{stress centrality},
introduced by \citet{s-spcn-53}. Shimbel assumes that the number of
shortest paths containing a node \(u\) is an estimate for the amount of
``stress'' the node has to sustain in a network. The more shortest paths
run through a node the more central it is. Formally, stress centrality
is defined as \[
c_{stress}(u)=\sum\limits_{s,t \in V} \sigma(s,t\lvert u),
\] where \(\sigma(s,t\lvert u)\) is the number of shortest paths from
\(s\) to \(t\) passing through \(u\). Instead of the absolute number of
shortest paths, betweenness centrality quantifies their relative number.
The relative count is given by \[
\delta(s,t\lvert u)= \frac{\sigma(s,t\lvert u)}{\sigma(s,t)},
\] where \(\sigma(s,t)\) is the total number of shortest paths
connecting \(s\) and \(t\). The expression \(\delta(s,t\lvert u)\) can
be interpreted as the extent to which \(u\) controls the communication
between \(s\) and \(t\). Betweenness is then defined as \[
c_b(u) = \sum\limits_{s,t \in V} \delta(s,t\lvert u)
\] The interpretation of betweenness is not only restricted to
communication. More generally, betweenness quantifies the influence of
vertices on the trans- fer of items or information through the network
with the assumption that it follows shortest paths. Many different
variants of shortest path betweenness have been proposed to incorporate
additional assumptions, e.g.~the specific lo- cation of a vertex \(u\)
on a shortest \(s,t\)-path or its length. Some of these variants are
given in the following table.

\begin{longtable}[]{@{}ll@{}}
\toprule
\endhead
\begin{minipage}[t]{0.81\columnwidth}\raggedright
proximal source\strut
\end{minipage} & \begin{minipage}[t]{0.13\columnwidth}\raggedright
\(c_{bs}(u) = \sum\limits_{s,t \in V} \delta(s,t\lvert u)\cdot A_{su}\)\strut
\end{minipage}\tabularnewline
\begin{minipage}[t]{0.81\columnwidth}\raggedright
proximal target\strut
\end{minipage} & \begin{minipage}[t]{0.13\columnwidth}\raggedright
\(c_{bt}(u) = \sum\limits_{s,t \in V} \delta(s,t\lvert u)\cdot A_{ut}\)\strut
\end{minipage}\tabularnewline
\begin{minipage}[t]{0.81\columnwidth}\raggedright
k-bounded distance\strut
\end{minipage} & \begin{minipage}[t]{0.13\columnwidth}\raggedright
\(c_{bk}(u) = \sum\limits_{dist(s,t)\leq k} \delta(s,t\lvert u)\)\strut
\end{minipage}\tabularnewline
\begin{minipage}[t]{0.81\columnwidth}\raggedright
length-scaled\strut
\end{minipage} & \begin{minipage}[t]{0.13\columnwidth}\raggedright
\(c_{bd}(u) = \sum\limits_{s,t \in V} \frac{\delta(s,t\lvert u)}{dist(s,t)}\)\strut
\end{minipage}\tabularnewline
\begin{minipage}[t]{0.81\columnwidth}\raggedright
linearly-scaled\strut
\end{minipage} & \begin{minipage}[t]{0.13\columnwidth}\raggedright
\(c_{bl}(u) = \sum\limits_{s,t \in V} \delta(s,t\lvert u)\frac{dist(s,u)}{dist(s,t)}\)\strut
\end{minipage}\tabularnewline
\bottomrule
\end{longtable}

Details on these variants can be found in \citet{b-vsbctgc-08}. Other
variants of the betweenness concept rely on different assumptions of
transfer in networks besides shortest paths.

A measure based on network flow was defined by \citet{fbw-cvgmbbnf-91}.
The authors assume information as a flow process and assign to each edge
a non-negative value representing the maximum amount of information that
can be passed between the endpoints. The aim is then to measure the
extent to which the maximum flow between two vertices \(s\) and \(t\)
depends on a vertex \(u\). Denote by \(f(s,t)\) the maximum
\((s,t)\)-flow w.r.t. constraints imposed by edge capacities and the
amount of flow which must go through \(u\) by \(f( s, t \lvert u )\) .
Similar to shortest path betweenness, \emph{flow betweenness} is then
defined as \[
c_f(u)=\sum\limits_{s,t \in V} \frac{f(s,t\lvert u)}{f(s,t)}.
\] The index was introduced as a betweenness variant for weighted
networks but it can easily applied to unweighted ones. In the case of
simple undirected and unweighted networks, the maximum \((s, t)\)-flow
is equivalent to the number of edge disjoint \((s,t)\)-paths and
\(f(s,t|u)\) is the number of paths \(u\) lies on.

A variant based on walks instead of paths was proposed by
\citet{n-mbcbrw-05}. His \emph{random walk betweenness} calculates the
expected number of times a random \((s,t)\)-walk passes through a vertex
\(u\), averaged over all \(s\) and \(t\). Newman shows, that his variant
of betweenness can also be calculated with a current-flow analogy by
viewing a graph as an electrical network. Random walk betweenness is
then equiva- lent to the amount of current that flows through \(u\). The
index is thus also known as \emph{current flow betweenness}. Details and
formal definitions of his versions can be found in the literature
\citep{n-mbcbrw-05, bf-cmbcf-05}.

Two variants based on the \emph{randomzed shortest path} (RSP) framework
\citep{ysms-fdmngbscd-08, safy-rsptrm-09, kss-dtrspcgnd-14} were
proposed by \citet{klss-tbcmbrsp-16}. The variants are refered to as
\emph{simple RSP betweenness} and \emph{RSP net betweenness}. The
derivation of both indices is rather involved and goes beyond the scope
of this tutorial. The interested reader should consult the original work
for details. One aspect worth mentioning, though, is that both variants
include a tuning parameter \(\beta\). Both variants converge to the
traditional version of betweennes for \(\beta \to \infty\). For
\(\beta \to 0\), simple RSP betweenness converges to the expected number
of visits to a node over all absorbing walks with respect to the
unbiased random walk probabilities. This means for undirected networks,
that the index converges to degree. RSP net betweenness, on the other
hand, converges to Newmann's random walk betweenness.

All variants of betweenness can be described in a more general form
considering a flow of information analogy. Depending on the assumption
of how information is `flowing' between \(s\) and \(t\), the set
\(P(s, t)\) contains all possible information channels to transmit the
piece of information. This set might contain all shortest
\(( s, t )\)-paths if the information has to be trans- mitted as fast as
possible or all random \((s, t)\)-walks when the delivery time does not
matter. In principle, any kind of trajectory on a graph can be thought
of as an information channel. The set \(P( s, t \lvert u )\) contains
all information channels where the vertex \(u\) is in a position to
control the information flow. For shortest path betweenness, \(u\) is in
a controlling position if it is part of an information channel and for
proximal target betweenness if it presents the information to the target
\(t\). In the former case \(P( s, t \lvert u )\) comprises all elements
of \(P(s, t)\) that contain \(u\) as an intermediary and in the latter
all elements that contain the edge \((u,t)\) . Again, the position of
control could be defined as any location on a trajectory. A measure of
relative betweenness is then defined with aggregation rules over the two
specified sets, commonly the fraction of their cardinalities. This
fraction can also be weighted according to specified rules, e.g.~as in
length scaled betweenness. Aggregating over all possible sources and
targets, we can thus define a \emph{generic betweenness} index as \[
c_{bg}(u) = \sum\limits_{s,t \in V}\frac{\lvert P(s,t\lvert u) \rvert}{\lvert P(s,t) \rvert} \cdot w(s,t).
\] Thus, many other variants are possible, for instance also
\(k\)-betweenness mentioned by \citet{be-gpc-06}, where \(P( s, t )\) is
the set of all \(( s, t )\)-paths of length at most \(k\).

There exist many more betweenness-like indices that where not covered
here, but are listed below in no particular order:

\begin{itemize}
\tightlist
\item
  \emph{communicability betweenness} based on the matrix exponential
  \citep{ehh-cbcn-09}
\item
  \(\alpha\) and \(\beta\) betweenness, which are closely related to
  current flow betweenness \citep{alms-acfbc-13, amt-bcfcwn-15},
\item
  \emph{ranking betweenness}, which combines betweenness with the idea
  of PageRank \citep{aotv-nbcmbarnn-14}
\item
  range-limited forms of betweenness \citep{elct-rcmcn-12}
\item
  \emph{bridgeness} \citep{jmkvvjcmf-dgbn-16}
\item
  \emph{super mediator} \citep{skom-sm-ncmniidsn-16}
\item
  \emph{BridgeRank} \citep{sam-bnfcmblsn-18}
\end{itemize}

\hypertarget{closeness-and-variants}{%
\section{Closeness (and variants)}\label{closeness-and-variants}}

\hypertarget{eigenvector-and-variants}{%
\section{Eigenvector (and variants)}\label{eigenvector-and-variants}}

\hypertarget{others}{%
\section{Others}\label{others}}

\bibliography{Indices.bib,Theory.bib}


\end{document}
